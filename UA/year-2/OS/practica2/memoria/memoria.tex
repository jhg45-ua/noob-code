\documentclass[11pt,a4paper]{article}

% ============================================================
% PAQUETES BÁSICOS
% ============================================================
\usepackage{fontspec}
\setmainfont{Myriad Pro} % Fuente principal
\usepackage[spanish]{babel}
\usepackage{graphicx}
\usepackage{geometry}
\usepackage{fancyhdr}
\usepackage{titlesec}
\usepackage{xcolor}
\usepackage{hyperref}
\usepackage{amsmath}      % Para ecuaciones matemáticas
\usepackage{float}        % Para mejor control de posición de figuras
\usepackage{caption}      % Para personalizar captions
\usepackage{subcaption}   % Para subfiguras si se necesitan
\usepackage{microtype}     % Mejora el espaciado y reduce overfull

% ============================================================
% CONFIGURACIONES BÁSICAS
% ============================================================

% Configuración de márgenes
\geometry{
    top=2.5cm,
    bottom=2.5cm,
    left=2cm,
    right=2cm,
    headheight=14.5pt
}

% Configuración de enlaces
\hypersetup{
    colorlinks=true,
    linkcolor=blue,
    filecolor=magenta,
    urlcolor=cyan,
    pdftitle={Memoria Práctica},
    pdfpagemode=FullScreen,
}

% Configuración de encabezados y pies de página
\pagestyle{fancy}
\fancyhf{}
\fancyhead[L]{Asignatura - Nº Práctica}
\fancyhead[R]{Mes - Año}
\fancyfoot[R]{\thepage}
\fancyfoot[L]{Nombre del Estudiante}

% ============================================================
% CONFIGURACIONES ESPECÍFICAS POR ASIGNATURA
% ============================================================

% --- [SISTEMAS OPERATIVOS] Configuración de listings para código ---
\usepackage{listings}

% Configuración de colores para el código
\definecolor{codegreen}{rgb}{0,0.6,0}
\definecolor{codegray}{rgb}{0.5,0.5,0.5}
\definecolor{codepurple}{rgb}{0.58,0,0.82}
\definecolor{backcolour}{rgb}{0.95,0.95,0.92}
\definecolor{codeorange}{rgb}{0.8,0.4,0}
\definecolor{framecolor}{rgb}{0.7,0.7,0.7}

% Configuración de listings para código C
\lstdefinestyle{mystyle}{
    backgroundcolor=\color{backcolour},   
    commentstyle=\color{codegreen}\itshape,
    keywordstyle=\color{blue}\bfseries,
    numberstyle=\tiny\color{codegray},
    stringstyle=\color{codepurple},
    basicstyle=\ttfamily\small,
    breakatwhitespace=false,         
    breaklines=true,                 
    captionpos=b,                    
    keepspaces=true,                 
    numbers=left,                    
    numbersep=8pt,
    showspaces=false,                
    showstringspaces=false,
    showtabs=false,                  
    tabsize=4,
    frame=single,
    frameround=tttt,
    rulecolor=\color{framecolor},
    framesep=4pt,
    xleftmargin=15pt,
    xrightmargin=5pt,
    language=C,
    extendedchars=true,
    inputencoding=utf8,
    escapeinside={(*@}{@*)},
    morecomment=[l][\color{codeorange}]{\#},
    columns=flexible,
    aboveskip=15pt,
    belowskip=10pt,
    literate={á}{{\'a}}1 {é}{{\'e}}1 {í}{{\'i}}1 {ó}{{\'o}}1 {ú}{{\'u}}1
             {Á}{{\'A}}1 {É}{{\'E}}1 {Í}{{\'I}}1 {Ó}{{\'O}}1 {Ú}{{\'U}}1
             {ñ}{{\~n}}1 {Ñ}{{\~N}}1 {¿}{{?`}}1 {¡}{{!`}}1
}

\lstset{style=mystyle}

% Configuración para el caption de los listings
\DeclareCaptionFont{white}{\color{white}}
\DeclareCaptionFormat{listing}{\colorbox{gray}{\parbox{\dimexpr\textwidth-2\fboxsep\relax}{#1#2#3}}}
\captionsetup[lstlisting]{format=listing, labelfont=white, textfont=white, singlelinecheck=false, margin=0pt, font={bf,footnotesize}}
% --- FIN [SISTEMAS OPERATIVOS] ---

\begin{document}

% --- Portada y índice sin numeración ---
\pagenumbering{gobble} % no mostrar números de página en preliminares

% Página de portada - Imagen ocupando la mayor parte de la página pero un poco reducida
\newgeometry{margin=0cm} % quitar márgenes solo para la portada
\thispagestyle{empty}
% uso de makebox + raisebox para centrar horizontal y verticalmente
\noindent\raisebox{0pt}[\paperheight][0pt]{%
  \makebox[\paperwidth][c]{%
    \includegraphics[width=\paperwidth,height=\paperheight,keepaspectratio]{portada.png}%
  }
}
\restoregeometry

\newpage

% Tabla de contenidos
\tableofcontents
\newpage

% --- A partir de aquí numeración normal ---
\pagenumbering{arabic}   % cambiar a 1,2,3...
\setcounter{page}{1}     % empezar en 1
\pagestyle{fancy}        % restaurar encabezados/pies definidos en el preámbulo

% Inicio del documento
\section{Introducción}

Aquí va la introducción de la memoria. Este documento presenta el desarrollo y resultados de la práctica de Sistemas Operativos, donde se abordan 
diversos ejercicios relacionados con [completar según el tema de la práctica].

\section{Objetivos}

Los objetivos de esta práctica son:

\begin{itemize}
    \item Comprender los conceptos fundamentales de [completar]
    \item Implementar soluciones prácticas utilizando [completar]
    \item Analizar el comportamiento y rendimiento de [completar]
\end{itemize}

\section{Desarrollo}

\subsection{Ejercicio 1}

En este ejercicio se desarrolla [descripción del ejercicio]. A continuación se presenta el análisis y la implementación realizada.

Este ejercicio requiere una comprensión profunda de los conceptos fundamentales estudiados en clase, así como la capacidad de aplicarlos en situaciones prácticas.

El desarrollo de este ejercicio implica varios pasos importantes que deben seguirse cuidadosamente. Primero, es necesario analizar el problema planteado
y comprender todos sus requisitos. Luego, se debe diseñar una solución apropiada que cumpla con todos los criterios especificados.

\section{Código Fuente}

A continuación se presenta el código implementado para los ejercicios:

\subsection{Código del Ejercicio 1}

\begin{lstlisting}[caption={Implementación del Ejercicio 1}, label={lst:ejercicio1}]
#include <stdio.h>
#include <stdlib.h>

int main() {
    // Código de ejemplo - reemplazar con tu implementación
    printf("Ejercicio 1: Hola Mundo\n");
    
    // Tu código aquí
    
    return 0;
}
\end{lstlisting}

\section{Conclusiones}

Tras la realización de esta práctica, se pueden extraer las siguientes conclusiones:

\begin{itemize}
    \item Conclusión 1
    \item Conclusión 2
    \item Conclusión 3
\end{itemize}

\section{Bibliografía}

\begin{itemize}
  \item Referencia 1: [Autor]. [Título]. [Editorial], [Año].
  \item Referencia 2: [Autor]. [Título]. [Editorial], [Año].
  \item Documentación oficial de [herramienta/lenguaje utilizado]
\end{itemize}
  
\end{document}