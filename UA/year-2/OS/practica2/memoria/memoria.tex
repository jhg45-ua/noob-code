\documentclass[11pt,a4paper]{article}

% ============================================================
% PAQUETES BÁSICOS
% ============================================================
\usepackage{fontspec}
\setmainfont{Myriad Pro} % Fuente principal
\usepackage[spanish]{babel}
\usepackage{graphicx}
\usepackage{geometry}
\usepackage{fancyhdr}
\usepackage{titlesec}
\usepackage{xcolor}
\usepackage{hyperref}
\usepackage{amsmath}      % Para ecuaciones matemáticas
\usepackage{float}        % Para mejor control de posición de figuras
\usepackage{caption}      % Para personalizar captions
\usepackage{subcaption}   % Para subfiguras si se necesitan
\usepackage{microtype}     % Mejora el espaciado y reduce overfull

% ============================================================
% CONFIGURACIONES BÁSICAS
% ============================================================

% Configuración de márgenes
\geometry{
    top=2.5cm,
    bottom=2.5cm,
    left=2cm,
    right=2cm,
    headheight=14.5pt
}

% Configuración de enlaces
\hypersetup{
    colorlinks=true,
    linkcolor=blue,
    filecolor=magenta,
    urlcolor=cyan,
    pdftitle={Memoria Práctica},
    pdfpagemode=FullScreen,
}

% Configuración de encabezados y pies de página
\pagestyle{fancy}
\fancyhf{}
\fancyhead[L]{Asignatura - Nº Práctica}
\fancyhead[R]{Mes - Año}
\fancyfoot[R]{\thepage}
\fancyfoot[L]{Nombre del Estudiante}

% ============================================================
% CONFIGURACIONES ESPECÍFICAS POR ASIGNATURA
% ============================================================

% --- [SISTEMAS OPERATIVOS] Configuración de listings para código ---
\usepackage{listings}

% Configuración de colores para el código
\definecolor{codegreen}{rgb}{0,0.6,0}
\definecolor{codegray}{rgb}{0.5,0.5,0.5}
\definecolor{codepurple}{rgb}{0.58,0,0.82}
\definecolor{backcolour}{rgb}{0.95,0.95,0.92}
\definecolor{codeorange}{rgb}{0.8,0.4,0}
\definecolor{framecolor}{rgb}{0.7,0.7,0.7}

% Configuración de listings para código C
\lstdefinestyle{mystyle}{
    backgroundcolor=\color{backcolour},   
    commentstyle=\color{codegreen}\itshape,
    keywordstyle=\color{blue}\bfseries,
    numberstyle=\tiny\color{codegray},
    stringstyle=\color{codepurple},
    basicstyle=\ttfamily\small,
    breakatwhitespace=false,         
    breaklines=true,                 
    captionpos=b,                    
    keepspaces=true,                 
    numbers=left,                    
    numbersep=8pt,
    showspaces=false,                
    showstringspaces=false,
    showtabs=false,                  
    tabsize=4,
    frame=single,
    frameround=tttt,
    rulecolor=\color{framecolor},
    framesep=4pt,
    xleftmargin=15pt,
    xrightmargin=5pt,
    language=C,
    extendedchars=true,
    inputencoding=utf8,
    escapeinside={(*@}{@*)},
    morecomment=[l][\color{codeorange}]{\#},
    columns=flexible,
    aboveskip=15pt,
    belowskip=10pt,
    literate={á}{{\'a}}1 {é}{{\'e}}1 {í}{{\'i}}1 {ó}{{\'o}}1 {ú}{{\'u}}1
             {Á}{{\'A}}1 {É}{{\'E}}1 {Í}{{\'I}}1 {Ó}{{\'O}}1 {Ú}{{\'U}}1
             {ñ}{{\~n}}1 {Ñ}{{\~N}}1 {¿}{{?`}}1 {¡}{{!`}}1
}

\lstset{style=mystyle}

% Configuración para el caption de los listings
\DeclareCaptionFont{white}{\color{white}}
\DeclareCaptionFormat{listing}{\colorbox{gray}{\parbox{\dimexpr\textwidth-2\fboxsep\relax}{#1#2#3}}}
\captionsetup[lstlisting]{format=listing, labelfont=white, textfont=white, singlelinecheck=false, margin=0pt, font={bf,footnotesize}}

\newcommand{\command}[1]{{\ttfamily\textcolor{orange!80!black}{\detokenize{#1}}}}
% --- FIN [SISTEMAS OPERATIVOS] ---

\begin{document}

% --- Portada y índice sin numeración ---
\pagenumbering{gobble} % no mostrar números de página en preliminares

% Página de portada - Imagen ocupando la mayor parte de la página pero un poco reducida
\newgeometry{margin=0cm} % quitar márgenes solo para la portada
\thispagestyle{empty}
% uso de makebox + raisebox para centrar horizontal y verticalmente
\noindent\raisebox{0pt}[\paperheight][0pt]{%
  \makebox[\paperwidth][c]{%
    \includegraphics[width=\paperwidth,height=\paperheight,keepaspectratio]{portada.png}%
  }
}
\restoregeometry

\newpage

% Tabla de contenidos
\tableofcontents
\newpage

% --- A partir de aquí numeración normal ---
\pagenumbering{arabic}   % cambiar a 1,2,3...
\setcounter{page}{1}     % empezar en 1
\pagestyle{fancy}        % restaurar encabezados/pies definidos en el preámbulo

% Inicio del documento
\section{Introducción}

Aquí va la introducción de la memoria. Este documento presenta el desarrollo y resultados de la práctica de Sistemas Operativos, donde se abordan 
diversos ejercicios relacionados con [completar según el tema de la práctica].

\section{Objetivos}

Los objetivos de esta práctica son:

\begin{itemize}
    \item Comprender los conceptos fundamentales de [completar]
    \item Implementar soluciones prácticas utilizando [completar]
    \item Analizar el comportamiento y rendimiento de [completar]
\end{itemize}

\section{Desarrollo}

\subsection{Ejercicio 1}

En este ejercicio se nos pide desarrollar una aplicacion cliente-servidor para poder transferrir un archivo desde el servidor al cliente utilizando 
sockets en C. Usaremos una version de la pagina de inicio de google como archivo de prueba.


El servidor se encargara de escuchar las conexiones entrantes y un proceso hijo dentro del mismo servidor se encargara de enviar el archivo al cliente 
una vez este se haya conectado. El archivo se enviara en bloques configurados por la macro \command{BUFFER\_SIZE}. El servidor estara constantemente escuchando
nuevas conexiones con un maximo de 5 conexiones en cola.


Mientras tanto, el cliente se encargara de conectarse al servidor y recibir el archivo en bloques, escribiendo cada bloque recibido en un archivo local. Una vez 
completado abrira el archivo en el navegador por defecto del sistema.

\subsection{Ejercicio 2}
El ejercicio 2 busca implementar mediante el simulador jBACI el problema del productor-consumidor utilizando semáforos para la sincronización entre procesos. Para este
ejercicio he reproducido el ejemplo de la transparencia vista en clase, adaptándolo al entorno de jBACI y añadiendo comentarios explicativos en el código para mayor 
claridad.

\section{Código Fuente}

A continuación se presenta el código implementado para los ejercicios:

\subsection{Código del Ejercicio 1}
Empezando con la parte del servidor, el siguiente codigo muestra la configuracion del socket, del servidor y su enlaze:

\begin{lstlisting}[caption={Configuracion del servidor}, label={lst:ejercicio1_1}]
// Creación del socket, se gestiona el error si no se puede crear
sockfd = socket(AF_INET, SOCK_STREAM, 0);
if (sockfd == -1) {
  fprintf(stderr, "Error al crear el socket");
  exit(EXIT_FAILURE);
}
printf("Socket creado con exito\n");

// Configuración de la estructura sockaddr_in
serverAddr.sin_port = htons(DEFAULT_PORT); // puerto del servidor
serverAddr.sin_family = AF_INET; // familia de direcciones IPv4
serverAddr.sin_addr.s_addr = INADDR_ANY; // aceptar conexiones en cualquier interfaz de red

// Enlazar el socket a la dirección y puerto especificados, con gestión de errores
if (bind(sockfd, (struct sockaddr *)&serverAddr, sizeof(serverAddr)) != 0) {
  fprintf(stderr, "Error al enlazar el socket");
  exit(EXIT_FAILURE);
}
\end{lstlisting}

El \command{sin_port} se configura con la función \command{htons} para convertir el número de puerto al formato de red adecuado, \command{sin_family} 
se establece en \command{AF_INET} para indicar que se utilizará IPv4, y \command{sin_addr.s_addr} se configura con \command{INADDR_ANY} para aceptar 
conexiones en cualquier interfaz de red disponible.

Ahora el servidor se pone a escuchar conexiones entrantes y acepta una conexion cuando llega una nueva:
\begin{lstlisting}[caption={El servidor acepta conexiones}, label={lst:ejercicio1_2}]
// Poner el socket en modo escucha para aceptar conexiones entrantes, maximo 5 conexiones en cola
listen(sockfd, 5);

printf("Esperando nuevas conexiones...\n");

// Bucle infinito para aceptar y manejar conexiones entrantes
while(1) {
  // Aceptar una nueva conexión de cliente
  size = sizeof(clientAddr);
  clientSocketfd = accept(sockfd, (struct sockaddr *)&clientAddr, &size);
  if (clientSocketfd == -1) {
    fprintf(stderr, "Error aceptando la conexion\n");
    continue;
  }

  printf("Conexion aceptada!!!\n");
  ....
}
\end{lstlisting}

Luego de aceptar la conexion, se crea un proceso hijo para manejar la transferencia del archivo al cliente, y se envia el archivo en bloques:
\begin{lstlisting}[caption={Envio del archivo en bloques}, label={lst:ejercicio1_3}]
//Leer y enviar el archivo por trozos en base al tamaño del buffer
while((bytesRead = read(filefd, buffer, sizeof(buffer))) > 0) {
    bytesSent = write(clientSocketfd, buffer, bytesRead);
    if (bytesSent == -1) {
        fprintf(stderr, "Error en la transferencia del archivo\n");
        break;
    }
}
\end{lstlisting}

Ya despues de eso simplemente se cierra el archivo y el socket del cliente en el proceso hijo, y el servidor vuelve a esperar nuevas conexiones.
\section{Conclusiones}

Tras la realización de esta práctica, se pueden extraer las siguientes conclusiones:

\begin{itemize}
    \item Conclusión 1
    \item Conclusión 2
    \item Conclusión 3
\end{itemize}

\section{Bibliografía}

\begin{itemize}
  \item Referencia 1: [Autor]. [Título]. [Editorial], [Año].
  \item Referencia 2: [Autor]. [Título]. [Editorial], [Año].
  \item Documentación oficial de [herramienta/lenguaje utilizado]
\end{itemize}
  
\end{document}